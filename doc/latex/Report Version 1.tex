\documentclass[11pt]{article}
\usepackage{geometry}                
\geometry{letterpaper}                   

\usepackage{graphicx}
\usepackage{amssymb}
\usepackage{epstopdf}
\usepackage{natbib}
\usepackage{amssymb, amsmath}
\DeclareGraphicsRule{.tif}{png}{.png}{`convert #1 `dirname #1`/`basename #1 .tif`.png}

%\title{Title}
%\author{Name 1, Name 2}
%\date{date} 

\begin{document}



\thispagestyle{empty}

\begin{center}
\includegraphics[width=5cm]{ETHlogo.eps}

\bigskip


\bigskip


\bigskip


\LARGE{ 	Lecture with Computer Exercises:\\ }
\LARGE{ Modelling and Simulating Social Systems with MATLAB\\}

\bigskip

\bigskip

\small{Project Report}\\

\bigskip

\bigskip

\bigskip

\bigskip


\begin{tabular}{|c|}
\hline
\\
\textbf{\LARGE{Solving the Travelling Salesman Problem}}\\
\textbf{\LARGE{by Using an Artificial Ant Colony}}\\
\\
\hline
\end{tabular}
\bigskip

\bigskip

\bigskip

\LARGE{Raphaela Wagner \& Giandrin Barandun}



\bigskip

\bigskip

\bigskip

\bigskip

\bigskip

\bigskip

\bigskip

\bigskip

Zurich\\
May 2014\\

\end{center}



\newpage

%%%%%%%%%%%%%%%%%%%%%%%%%%%%%%%%%%%%%%%%%%%%%%%%%

\newpage
\section*{Agreement for free-download}
\bigskip


\bigskip


\large We hereby agree to make our source code for this project freely available for download from the web pages of the SOMS chair. Furthermore, we assure that all source code is written by ourselves and is not violating any copyright restrictions.

\begin{center}

\bigskip


\bigskip


\begin{tabular}{@{}p{3.3cm}@{}p{6cm}@{}@{}p{6cm}@{}}
\begin{minipage}{3cm}

\end{minipage}
&
\begin{minipage}{6cm}
\vspace{2mm} \large Raphaela Wagner

 \vspace{\baselineskip}

\end{minipage}
&
\begin{minipage}{6cm}

\large Giandrin Barandun

\end{minipage}
\end{tabular}


\end{center}
\newpage

%%%%%%%%%%%%%%%%%%%%%%%%%%%%%%%%%%%%%%%



% IMPORTANT
% you MUST include the ETH declaration of originality here; it is available for download on the course website or at http://www.ethz.ch/faculty/exams/plagiarism/index_EN; it can be printed as pdf and should be filled out in handwriting


%%%%%%%%%% Table of content %%%%%%%%%%%%%%%%%

\tableofcontents

\newpage

%%%%%%%%%%%%%%%%%%%%%%%%%%%%%%%%%%%%%%%



\section{Abstract}

\section{Individual contributions}
\subsection{Raphaela Wagner}
With the aim of achieving a good model for solving the travelling salesman problem by the use of artificial ants Raphaela helped the group understanding the underlying paper and the included model. She contributed a great amount of explanations and ideas how to approach the whole project. \\
In a second step she took care of how to implement the raw data from the TSP-library into MATLAB and transform it to a usable form. Further more she coded the functions "eta.m", "global\_pheromene\_update.m", "test\_funktionen.m" and "update.m" and helped improving and correcting the main program. \\
After the code was written she did a lot of testing with different problem sets and compared the solution of the program to known solutions. 

When the group got stuck and did not see a way out of a specific problem she was the one to bring along a hot chocolate and cheer the group up again.

\subsection{Giandrin Barandun}
The paper which is thought to be reconstructed on the following pages was selected and suggested to the group by Giandrin and during the whole process he tried to have some influence on the project with his wide technical understanding of the problem. \\
The codes for the functions "prob\_dist.m", "calc\_Lnn.m", "choose\_city.m", "main\_initialize\_system.m", "coordinates.m" and "calc\_dist.m" are his contributions as well as the collaboration on the main program. He searched the internet for known TS-problems and their solutions and put all data in a readable form. A lion's share for the program working at the end was his bug fixing in all the functions and programs and combining them to the running model.

\section{Introduction and Motivations}
%Introduction and Motivations

Observing crawling ants how they manage to find a shortest path from a food source to their nest arises the question of how to model such a biological phenomena. It is known that the way ants organize their transporting system is based on a secreted chemical called pheromone. While ants move on a track they deposit a certain amount of pheromone. Since real ants prefer choosing lines of a high pheromone concentration, this messenger ensures that ants follow their members on a certain trail.
To illustrate the effect of pheromone on an ant trail consider Figure \ref{fig:ants}. Real ants follow a path between a food source and their nest (Fig. \ref{fig:ants} A). Placing an obstacle on the trail forces the ants to find a way of restoring the interrupted track (Fig. \ref{fig:ants} B). One expects half of the ants to turn right and half of them to turn left. In the beginning both ways around the obstacle are enriched with approximately the same amount of pheromone (Fig. \ref{fig:ants} C). Since ants that have chosen the shortest path need less time to pass by the obstacle the number of ants per time is bigger compared to those who have chosen the longer path. Consequently the shorter path contains a higher concentration of deposited pheromone than the longer one. This follows from the assumption that all ants secrete the same amount of pheromone and move approximately at the same speed. \\
After a certain time more ants prefer the shorter path containing more pheromone until the longer one is completely neglected to circumvent the obstacle (Fig. \ref{fig:ants} D).

\begin{figure}[h!]
\begin{center}
\includegraphics[width=11cm, height= 6 cm]{ants}
\caption{(A): Ants following a trail between food source and their nest. (B): An obstacle is placed to interrupt the track of the ants. (C): The column of ants splits into two groups each choosing a different way to circumvent the obstacle. (D): Due to the higher concentration of pheromone all ants have chosen the shortest path.}
\label{fig:ants}
\end{center}
\end{figure}

Consulting the literature \cite{paper} one finds an interesting paper which models the ant colony system (ACS) using a traveling sales man problem (TSP). Artificial ants, also called agents are successively moving on a TSP graph between different cities. In the course of this they are following the constraint to visit each city once and return to their starting point. After all ants have completed their tour, the shortest one is rewarded by increasing the weight of the according tracks. This corresponds to a higher concentration of pheromone on the chosen tour.
The goal of this project is to implement the given model (see chapter \ref{sec:model} on page \pageref{sec:model}) from \cite{paper} and to calculate the shortest tour for different city environments. Those are obtained from the TSPLIB (\emph{http://www.iwr.uni-heidelberg.de/groups/comopt/software/TSPLIB95/tsp/}) and correspond to the data used in the reference paper \cite{paper}. The aim is to figure out whether our code is able to produce the same length of the shortest tour or not. Next to that, the variation of parameters in the model is analysed. Precisely, these simulations try to answer questions like: How does the shortest tour length depend on the rewarding, i.e. on the amount of pheromone deposited? How fast is the decay in the shortest tour as a function of completed rounds? Moreover, the influence of the ACS size on the time needed to find the shortest tour is investigated.
\section{Description of the Model}
In this model a number of $k$ ants is sent on a network of $m$ cities with every ant starting at the same city. The next ant only starts when the previous ant has finished its tour which means there never are two ants in the network. \\
\subsection{Choosing a City}
Before moving on to the next city an ant has to decide where it wants to go. For this purpose it chooses randomly a number $q$ between zero and one and if this number is smaller or equal than a certain parameter $q_0$ ($q \leqslant q_0$) looks for the city $s$ which fulfils the following formula:
\begin{equation}
s = \text{arg max}\{[\tau (r,u)] \cdot [\eta (r,u)]^{\beta}\}, \hspace{2cm} u \not\in M_k.
\label{eq:qsmallerq0}
\end{equation}
The current city where the ant stays is denoted by $r$ and only cities can be chosen which are not yet in the ants memory $M_k$ which means the ant has not visited these cities. The matrix $\tau$ stores the information about the amount of pheromone on the edge between city $r$ and city $u$ and the function $\eta$ gives the inverse of the distance between the two cities. \\
If the random number is bigger than $q_0$ ($q > q_0$) then the ant randomly chooses one of the remaining unvisited cities and accepts or rejects it according to the probability $p_k$:
\begin{equation}
p_k (r,s) = \frac{[\tau (r,s)] \cdot [\eta (r,s)]^{\beta}}{\sum_{u \not\in M_k}[\tau (r,u)] \cdot [\eta (r,u)]^{\beta}}
\label{eq:prob}
\end{equation}
This probability basically contains the same formula of $\tau$ and $\eta$ as the one above but is now normalized with the sum over these relations of every unvisited city. One can clearly see that in the beginning the sum is big and the probabilities are small but favouring the edges with more pheromone and lower distance. At the end when only one city is left the sum equals the term in the nominator and the probability becomes one for the last remaining city.
\subsection{Moving Forward and Updating}
Once the city has been chosen the ant moves along the edge and the pheromone on the path is updated according to:
\begin{equation}
\tau (r,s) = (1-\alpha)\cdot \tau(r,s) + \alpha \tau_0
\label{eq:loctauupdate}
\end{equation}
The newly introduced parameters $\alpha$ and $\tau_0$ are explained in chapter \ref{sec:results}. This update reduces the amount of pheromone on the chosen edge and helps avoiding very strong edges which would be taken by all the ants. \\
At the time the first ant has finished the tour the second one can start while the first ant still keeps in mind the trajectory of its tour which means the sequence of the city it has visited and the length of its tour but deletes its memory such that it is ready for a new tour. When all ants have completed one tour the shortest one is rewarded with pheromone according to the formula:
\begin{equation}
\tau(r,s) = (1-\alpha)\cdot \tau(r,s) + \alpha \Delta \tau(r,s)
\label{eq:globalupdate}
\end{equation}
This update is intended to give the edges along the shortest path a little head start in the following round. With this step the first round is complete and the second round can start.

\subsection{Modifications}

In the described model above every following ant waited its predecessor to finish the tour before it started its own tour. In the paper on which this report is based on the model was slightly different. The agents do not wait for the previous ones but start moving all together. After the first ant has chosen the next city it sits there until all agents have finished their first step and then step number two can begin and the ants choose the third city to go to. \\
The difference to the model described before is that the edges get constantly locally updated and every agent feels the influence of the other agents currently being in the network. Whereas in the first model the second ant only sees the trace the first one left but not the ones of all agents still waiting at the first city and so on. \\
Another small modification was to increase the award for the edges along the shortest tour. This means to add to $\tau$ (equation \ref{eq:globalupdate}) a constant amount of pheromone for the global update.

\begin{equation}
\tau(r,s) = (1-\alpha)\cdot \tau(r,s) + \alpha \Delta \tau(r,s)\textbf{ + 0.1}
\label{eq:globalupdate2}
\end{equation}
\section{Implementation}
\subsection{Main program}
In the previous section the theoretical understanding of the model was tried to be imparted to the reader. Following in this section is an overview of how the model was implemented in MATLAB. \\
To start with a main program (\textit{main\_initialize\_system.m}) was written which reads the data of a specific TSP and puts it into an upper triangle matrix form which contains the distances between the cities as elements. Further more all parameters like number of rounds and number of contributing ants (agents) can be adjusted in this main program. At the end it activates the function \textit{main\_main.m} with the purpose to find the shortest route. \\
In the main function there are two for-loops, one for the number of rounds and one for the number of ants (agents). For every agent we have a memory of the visited cities (\textit{M\_k})and a trajectory vector which saves the sequence of how the cities where visited. In the first step every ants chooses the next city with help of the function \textit{choose\_city.m} which is described below. The city is added to the memory of the ant and the trajectory vector and the pheromone on the edge is reduced according to the formula on page 75 in the paper. As long as there are unvisited cities this procedure is repeated for the agents and at the end the way to the start city is updated (pheromone, trajectory) and the memory reset. Then the second agent starts his tour. \\
After every agent has finished his route the shortest path is detected and the edges along this route are rewarded with pheromone with help of the trajectory vector and the function \textit{global\_pheromone\_update.m}. At the end of this step the trajectory vector is reset for all ants and the second round can be initialized. After every round the shortest route of the current round is compared to the overall shortest route found until now and then kept or rejected depending on the result of the comparison. \\
When all rounds have been calculated the function gives the shortest path found in any of the rounds.

\subsection{Choose the Next City}

To choose which city the ant will go next two different methods are implemented according to the paper and one of them is selected based on a certain probability. \\
The first way to decide which city to go next is to optimize a function which depends on the amount of pheromone on the edge between the current city and the chosen one and its length. This method chooses the edge with the highest amount of pheromone and the shortest length whereas there are parameters to weight these two variables relatively to each other. \\
On the other hand a probability was assigned to every unvisited city again depending on the amount of pheromone and the length of the edge between the cities. Then a city was randomly chosen and accepted or rejected with its assigned probability.\\
At the end the function \textit{choose\_city.m} gives the number of the next city to visit back to the main function.
\section{Simulation Results and Discussion}

\section{Summary and Outlook}

\section{References}






\end{document}  



 
