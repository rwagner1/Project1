\documentclass[11pt]{article}
\usepackage{geometry}                
\geometry{letterpaper}                   

\usepackage{graphicx}
\usepackage{amssymb}
\usepackage{epstopdf}
\usepackage{natbib}
\usepackage{amssymb, amsmath}
\DeclareGraphicsRule{.tif}{png}{.png}{`convert #1 `dirname #1`/`basename #1 .tif`.png}

%\title{Title}
%\author{Name 1, Name 2}
%\date{date} 

\begin{document}



\thispagestyle{empty}

\begin{center}
\includegraphics[width=5cm]{ETHlogo.eps}

\bigskip


\bigskip


\bigskip


\LARGE{ 	Lecture with Computer Exercises:\\ }
\LARGE{ Modelling and Simulating Social Systems with MATLAB\\}

\bigskip

\bigskip

\small{Project Report}\\

\bigskip

\bigskip

\bigskip

\bigskip


\begin{tabular}{|c|}
\hline
\\
\textbf{\LARGE{Solving the Travelling Salesman Problem}}\\
\textbf{\LARGE{by Using an Artificial Ant Colony}}\\
\\
\hline
\end{tabular}
\bigskip

\bigskip

\bigskip

\LARGE{Raphaela Wagner \& Giandrin Barandun}



\bigskip

\bigskip

\bigskip

\bigskip

\bigskip

\bigskip

\bigskip

\bigskip

Zurich\\
May 2014\\

\end{center}



\newpage

%%%%%%%%%%%%%%%%%%%%%%%%%%%%%%%%%%%%%%%%%%%%%%%%%

\newpage
\section*{Agreement for free-download}
\bigskip


\bigskip


\large We hereby agree to make our source code for this project freely available for download from the web pages of the SOMS chair. Furthermore, we assure that all source code is written by ourselves and is not violating any copyright restrictions.

\begin{center}

\bigskip


\bigskip


\begin{tabular}{@{}p{3.3cm}@{}p{6cm}@{}@{}p{6cm}@{}}
\begin{minipage}{3cm}

\end{minipage}
&
\begin{minipage}{6cm}
\vspace{2mm} \large Raphaela Wagner

 \vspace{\baselineskip}

\end{minipage}
&
\begin{minipage}{6cm}

\large Giandrin Barandun

\end{minipage}
\end{tabular}


\end{center}
\newpage

%%%%%%%%%%%%%%%%%%%%%%%%%%%%%%%%%%%%%%%



% IMPORTANT
% you MUST include the ETH declaration of originality here; it is available for download on the course website or at http://www.ethz.ch/faculty/exams/plagiarism/index_EN; it can be printed as pdf and should be filled out in handwriting


%%%%%%%%%% Table of content %%%%%%%%%%%%%%%%%

\tableofcontents

\newpage

%%%%%%%%%%%%%%%%%%%%%%%%%%%%%%%%%%%%%%%



\section{Abstract}

\section{Individual contributions}
\subsection{Raphaela Wagner}
With the aim of achieving a good model for solving the travelling salesman problem by the use of artificial ants Raphaela helped the group understanding the underlying paper and the included model. She contributed a great amount of explanations and ideas how to approach the whole project. \\
In a second step she took care of how to implement the raw data from the TSP-library into MATLAB and transform it to a usable form. Further more she coded the functions "eta.m", "global_pheromene_update.m", "test_funktionen.m" and "update.m" and helped improving and correcting the main program. \\
After the code was written she did a lot of testing with different problem sets and compared the solution of the program to known solutions. 

When the group got stuck and did not see a way out of a specific problem she was the one to bring along a hot chocolate and cheer the group up again.

\subsection{Giandrin Barandun}
Giandrin tried to have some influence on the project with his wide technical understanding of the problem. The codes for the functions "prob_dist.m", "calc_Lnn.m", "choose_city.m", "main_initialize_system.m", "coordinates.m" and "calc_dist.m" are his contributions as well as the collaboration on the main program. He searched the internet for known TS-problems and their solutions and put all data in a readable form. A lion's share for the program working at

\section{Introduction and Motivations}

\section{Description of the Model}

\section{Implementation}

\section{Simulation Results and Discussion}

\section{Summary and Outlook}

\section{References}






\end{document}  



 
