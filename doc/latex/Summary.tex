%Summary and Outlook
The first aim of this project, namely the reproduction of the results from [1] was achieved. For two different city problems the shortest path averaged over ten or twenty trials and the best shortest tour ever reached correspond to the solutions known from [1], [2], [3]. Using two slightly different implementations of the same program do not show a significant difference in the obtained results.\\The second goal set was the dependency of the averaged shortest path on various parameters such as $\alpha$, $\beta$, $q_0$ and $\tau_\text{start}$. On one hand the analysis motivated chosen parameter values in [1] and on the other hand it manifested the independence of the shortest path (for example see \ref{fig:tauzushortestpath} in the appendix on page \pageref{fig:tauzushortestpath}). Moreover the obtained results revealed interesting characteristics of the model and helped for deeper understanding.\\
An investigation on the time the program needs to find the shortest path for various number of agents did not show a significant correlation. Therefore the impact of the agent number on the computation time would be an interesting topic to focus in a next project. Besides, for further investigations on TSP containing more nodes (cities) one should definitely optimize the MATLAB code for faster calculation.