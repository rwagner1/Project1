%Simulation Results and Discussion


\subsection{Length of shortest tour}

In a first attempt the dependency of the shortest tour length on different parameters is analysed. Thereby the number of completed rounds each agent finishes is varied. Moreover the value of the update rate $\alpha$ is optimized and the relative weighting of deposited pheromone and closeness is examined by changing the parameter $\beta$. \\

The shortest tour length is calculated for a 30 and 51 cities problem (called \emph{oliver30} and \emph{eil51} respectively) and averaged over 10 trials. The error bars in the following plot always refer to the standard deviation from those repeated measurements.\\The distances between the cities are determined using the given coordinates on a two dimensional euclidean plane. The distance $d_\text{i,j}$ between city $i$ and $j$ is then calculated by
\begin{equation}
d = \sqrt{(x_i-x_j)^2+(y_i-y_j)^2}.
\label{eq:dist}
\end{equation}
For \emph{oliver30} the distances are rounded to the next integer values, whereas in \emph{eil51} double distances are used. This leads to slightly different numbers for the shortest tour length.\\

The trajectory vector of the shortest tour is defined as the sequence of city numbers which yields the optimal tour length. For the city environment \emph{oliver30} the solution found in \cite{oli} is given by $[1~3~  4~  5~  6~  7~  8~  9~  10~  11~  12~  13~  14~  15~  16~  17~  18~  19~  20~  21~  22~  23~  25~  24~  26~  27~  28~  29~  30~  2]$. Using equation (\ref{eq:dist}) the total tour length is calculated to $shortest~path_\text{solution} = 420$. Comparing it to the sequence obtained from own simulations one observes a somewhat different sequence: $[1~2 ~3~  4~  5~  6~  7~  8~  9~  10~  11~  12~  13~  14~  15~  16~  17~  18~  19~  20~  21~  22~  23~  25~  24~  26~  27~  28~  29~  30~  1]$ where the agents don't move from city 30 to 2 but to 1. However, the length of this shortest path also results in $shortest~ path_\text{simulation}=420$. Obviously, there exist two solution to this symmetric TSP using rounded distances.\\

In \emph{eil51} the trajectory vector found using the model from \cite{paper} blabalbal. The length is balbalba
$\text{shortest path}_\text{solution} = 426$.

In Figure \ref{fig:roundspeil} and \ref{fig:roundspoli} the dependency of the shortest path length on the number of completed rounds is shown. As expected the average length decreases asymptotically towards the value of the optimal path. For both city environments the program achieved to reach the averaged shortest path known from literature \cite{paper} and \cite{oli}. In \emph{oliver30} $<$shortest path$>$ is in fact equal the best result found using 400 rounds, whereas in \emph{eil51} a lot more rounds would have been needed to achieve this. Anyhow, the best result obtained for \emph{eil} in a single run is $shortest path=429.48 \pm 1.86$. This shows that \\

\begin{table}[h!]
%\setlength{\tabcolsep}{5pt}
\renewcommand{\arraystretch}{1.2}
\center
\begin{tabular}{|c|c|c|c|c|}
	\hline
		&ACS average	\cite{paper} and \cite{oli} &		ACS best result \cite{paper} and \cite{oli}	& 		ACS average own simulations	&	ACS best result own simulations \\\hline
	
		\emph{eil51}	&  433.87	&	429 	& $433.7 \pm 3.2$ & 429.48  \%	\\\hline
		\emph{oli30}	&  (N/A)	  	&	420		& $420.5 \pm 0.5$ & 420		\%		\\\hline
		

		
\end{tabular}
\caption{Simulation results of two city environments compared to values from literature \cite{paper} and \cite{oli}.}
\label{tab:val}
\end{table}








In Figure \ref{fig:roundspeil} three models have been implemented and tested. The first (white circle dots) correspond exactly to the one described in \cite{paper} where all agents are moving at the same time step. The red squares represent the result obtained by a slightly modified model. The same mathematics has been used but the agents are exploring a new trail one by one. As can be seen, this change has little effect on the value of the shortest tour length and its error.\\ Adapting the pheromone update equations (\ref{eq:localtauupdate}) and (\ref{eq:globalupdate}) on page \pageref{sec:model} by adding a constant of 0.1, results in a steeper fall of the average shortest path. At 2000 rounds the final value does not yet reach the averaged shortest path length from \cite{paper}. One concludes that the added constant was chosen too high such that the global and local update only had a neglecting influence on $\tau(r,s)$.\\




\begin{figure}[h!]
\begin{center}
\includegraphics[width=11cm, height= 6 cm]{rounds_vs_shortestpath_eil}
\caption{Shortest tour length averaged over 10 runs for different numbers of rounds and 10 agents with the corresponding standard deviation as error bars. Thereby two slightly different implementations of the model from literature \cite{paper} are done (white circle dots and red squares). The green diamonds correspond to a model where the update formula $\tau(r,s)$ is changed. The values approximate the optimal solution for $\approx$ 1200 rounds. The two different lines represent the best shortest tour and the average value measured according to \cite{paper}. The parameters are chosen as follows $\alpha=0.1$, $\beta=2$,$q_0=0.9$,$tau_\text{start}=0.1$.}
\label{fig:roundspeil}
\end{center}
\end{figure}


\begin{figure}[h!]
\begin{center}
\includegraphics[width=11cm, height= 6 cm]{rounds_vs_shortestpath_oli}
\caption{Shortest tour length averaged over 10 runs for different numbers of rounds and 10 agents with the corresponding standard deviation as error bars. The value approximates the optimal solution for $\approx$ 400 rounds. For this simulation the model and its implementation are done in the same way as in \cite{paper}.}
\label{fig:roundspoli}
\end{center}
\end{figure}


Figure \ref{fig:alphasp} illustrates that the averaged shortest path length is optimized for $\alpha$ between 0.1 and 0.3. This is the region where the length of the tour and also the error of the shortest path is minimized. The reason why the shortest path is highest close zero and one, can be understood considering the pheromone update formulas (\ref{eq:localtauupdate}) and (\ref{eq:globalupdate}) on page \pageref{sec:model}. For $\alpha=0$ the local and global pheromone concentration stays constant, no update is made. Hence the cities are only favored by closeness and not by the pheromone concentration anymore. On the other hand, if $\alpha$ is close to one the amount of pheromone after the local update has changed a lot. Randomly chosen trails which deviate from the shortest path will be weighted too much. Therefore the optimal update rate $\alpha$ is expected to be closer to zero. 

\begin{figure}[h!]
\begin{center}
\includegraphics[width=11cm, height= 6 cm]{alpha_vs_shortestpath}
\caption{The update rate parameter $\alpha$ changes the averaged shortest tour length. The simulation is done for a fifty city problem using ten agents which all completed 2000 rounds. Other parameters are set to $beta=2$, $q_0=0.9$ and $tau_{\ref{start}}=0.1$. The values are averaged over ten runs. By setting $\alpha=0.1$ the length of the averaged shortest path is comparable to the value found in literature \cite{paper} and the change in pheromone update is held low.}
\label{fig:alphasp}
\end{center}
\end{figure}

For the parameter $\beta$ a similar analysis has been done for a thirty city environment and ten agents. The motivation for setting $\beta=2$ in \cite{paper} is barely understood from figure \ref{fig:betasp}. The decrease of the error for higher values of $\beta$ is explained by the fact that the criteria of closeness between the cities is weighted stronger than the pheromone concentration (see equation (\ref{eq:prob}) and (\ref{eq:qsmallerq0})). Thus, the exploration of new paths is suppressed and the deviation from the averaged path length is small. In the region of $\beta\approx2$ the bigger error bars indicate that the relative importance of pheromone and closeness is chosen such that the ants are forced to try new trails. The probability of getting stuck in a local minimum is reduced.

\begin{figure}[h!]
\begin{center}
\includegraphics[width=11cm, height= 6 cm]{beta_vs_shortestpath}
\caption{Variation of the parameter $\beta$ for the oliver30 using ten ants going 400 rounds averaged over 20 runs. The parameters are chosen as $\alpha=0.1$,$q_0=0.9$ and $tau_{\ref{start}}=0.1$.}
\label{fig:betasp}
\end{center}
\end{figure}



\subsection{Model adaptions}

