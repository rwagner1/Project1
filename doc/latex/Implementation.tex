\subsection{Main program}
In the previous section the theoretical understanding of the model was tried to be imparted to the reader. Following in this section is an overview of how the model was implemented in MATLAB. \\
To start with a main program (\textit{main\_initialize\_system.m}) was written which reads the data of a specific TSP and puts it into an upper triangle matrix form which contains the distances between the cities as elements. Further more all parameters like number of rounds and number of contributing ants (agents) can be adjusted in this main program. At the end it activates the function \textit{main\_main.m} with the purpose to find the shortest route. \\
In the main function there are two for-loops, one for the number of rounds and one for the number of ants (agents). For every agent we have a memory of the visited cities (\textit{M\_k})and a trajectory vector which saves the sequence of how the cities where visited. In the first step every ants chooses the next city with help of the function \textit{choose\_city.m} which is described below. The city is added to the memory of the ant and the trajectory vector and the pheromone on the edge is reduced according to the formula on page 75 in the paper. As long as there are unvisited cities this procedure is repeated for the agents and at the end the way to the start city is updated (pheromone, trajectory) and the memory reset. Then the second agent starts his tour. \\
After every agent has finished his route the shortest path is detected and the edges along this route are rewarded with pheromone with help of the trajectory vector and the function \textit{global\_pheromone\_update.m}. At the end of this step the trajectory vector is reset for all ants and the second round can be initialized. After every round the shortest route of the current round is compared to the overall shortest route found until now and then kept or rejected depending on the result of the comparison. \\
When all rounds have been calculated the function gives the shortest path found in any of the rounds.

\subsection{Choose the Next City}

To choose which city the ant will go next two different methods are implemented according to the paper and one of them is selected based on a certain probability. \\
The first way to decide which city to go next is to optimize a function which depends on the amount of pheromone on the edge between the current city and the chosen one and its length. This method chooses the edge with the highest amount of pheromone and the shortest length whereas there are parameters to weight these two variables relatively to each other. \\
On the other hand a probability was assigned to every unvisited city again depending on the amount of pheromone and the length of the edge between the cities. Then a city was randomly chosen and accepted or rejected with its assigned probability.\\
At the end the function \textit{choose\_city.m} gives the number of the next city to visit back to the main function.